\documentclass[11pt]{article}
\usepackage{acl2013}
\usepackage{times}
\usepackage{url}
\usepackage{latexsym}

\usepackage{covington}
\usepackage{graphicx}

\title{Improving word completion with idiolects and sociolects}

% anonymous for now
%\author{Wessel Stoop and Antal van den Bosch}

\author{First Author \\
  Affiliation / Address line 1 \\
  Affiliation / Address line 2 \\
  Affiliation / Address line 3 \\
  {\tt email@domain} \\\And
  Second Author \\
  Affiliation / Address line 1 \\
  Affiliation / Address line 2 \\
  Affiliation / Address line 3 \\
  {\tt email@domain} \\}

\date{}

\begin{document}
\maketitle

\begin{abstract} 
In this study we explore how word completion suggestions based on context-sensitive word prediction can be improved by incorporating typical language use by the author who uses the system in a social media context, as well as texts from the author's social context. First we show that our word prediction system performs better when it is tested on short messages by the same authors it has been trained on. By training on the language of earlier tweets, for some Twitter users up to 53\% of the keystrokes of new tweets can be saved. 
%These results can further be improved by training on more tweets, in particular on tweets produced by people related to the author somehow.
\end{abstract}

\section{Introduction}
Predicting what somebody is going to type in a text editor entails more than just knowledge of the language he or she uses; it is impossible to say what the next sentence of this text is going to be only on the basis of knowledge of English. \newcite{verberne12} show this also true for word prediction applications used in predictive editing (or word completion) systems. Predictive editing has become a regular and popular commodity in text editors on mobile devices and in query completion. These applications try to predict the word the user currently typing, or is going to type next, with the goal to reduce the number of keystrokes. According to Verberne {\it et al}., better results can be achieved if their system has been trained on texts from a similar domain, and thus already has some useful expectations of what is going to come. \newcite{vandenbosch11} takes another approach to solve the same problem: he makes the system learn 'on the fly' by adding a recency buffer. This means that a list of the $n$ most recent words is kept in memory. Whenever the word the user is keying in matches a word in this buffer, this word is predicted. So in the sentence \emph{They decided to go swimming because t...}, 'they' will be predicted because it was used recently.

Both approaches help to increase the number of keystrokes saved, but both approaches also have downsides: for the system by \newcite{verberne12} training texts in the same genre are needed, which might not be available, whereas the system by \newcite{vandenbosch11} ignores context information that it should weigh more intelligently. For example, while a context-sensitive text-prediction system will probably be able to predict 'to' for the sentence \emph{Tommy was going t...}, the one with the recency buffer will predict 'Tommy'.

In this paper two alternative approaches are suggested: using texts written by the user earlier as training material, and using texts from the author's social context as additional training material. The former emulates the sociolinguistic notion of ideolect, while the latter approximates the notion of sociolect. Using an idiolect (i.e. the language from only one person) instead of a general language model makes sense from a linguistic point of view as well; a small but growing body of literature suggests there are significant differences between the language of individual language users, on all levels of language \cite[among others]{mollin09,barlow10}. The same is likely to hold for the sociolect, the aggregate of language an individual language user is exposed to in his social communication.

The modeling of an ideolect can be seen as the relatively well-understood notion of personalization, while the second approach, to model sociolect, is novel. To investigate the usefulness of idiolect and sociolect modeling in text prediction, we will use tweets. Twitter provides an excellent data source for researching idiolect and sociolect, as Twitter in essence is a collection of language marked by author, as well as a network of authors. However, when trying to predict new tweets on the basis of earlier tweets, we have a practical and a theoretical problem. The practical problem is many of the datasets we can harvest per user will be relatively small. The theoretical problem is that the output of a particular person is only part of the idiolect; of course, people know much 'more language' than they produce. As an approximation of the language a person knows, it may make sense to harvest some of the the linguistic \emph{input} for a particular individual, the sociolect that this person is exposed to via social communication. We will try to solve both problems at once by also including the tweets of the 'friends' of the Twitter users investigated in the experiments described here. 

The operationalization of the notion of 'friends' or social contacts on social media such as Twitter is problematic and can only be approximated in an automatic processing setup. We will assume that all users who talk to each other at least six times symmetrically, i.e. three times from user A to B and three times from B to A, are friends. Of all input that could have been chosen for a particular Twitter user, we believe that the language of his orher friends is one of the best choices, as it is well known in the field of sociolinguistics that people to some extent align their language use in communication [[refs?]]. The language varieties that emerge by these alignment processes are called sociolects.

The questions we are trying to answer thus are:

\begin{enumerate}
\item Is it possible to do more accurate predictions if we train our language system on texts by the same author?
\item Is it possible to do more accurate predictions if we train our language system on texts by people related to the author in some way?
\end{enumerate}

\section{Methodology}

\subsection{Text prediction system: Xxx}

The software used for the text prediction experiments, which has the working title \emph{Xxx}\footnote{Name of system is anonymized}, consists of two types of modules: context-sensitive and context-insensitive. 

\textbf{Context-sensitive modules} are based on \emph{TiMBL}. TiMBL is an open source software package implementing several memory-based learning algorithms. For reasons of speed, TiMBL's IGTree was used. IGTree is an approximation of IB1-IG (which is k-nearest neighbour classification), but reduces the classification process to a much faster decision tree. We refer to \newcite{daelemans97} for details about IGTree.

Text prediction is thus treated as a classification task by these modules. First, Xxx transforms the training texts into word 4-grams. The first three words of each 4-gram are the features, the last one is the class. On the basis of these 4-grams an IGTree is created, henceforth referred to as the 'language model'. With this language model, new examples can be classified. Classifying a new example means that a 3-gram is given to TiMBL, and that TiMBL responds with a list of words that can possibly follow this 3-gram, along with confidence values. 

Xxx's job thus is collecting three words to the left of what the user is typing, and asking TiMBL to classify these three words. The resulting class, which is the word that will most likely follow what has been keyed in so far according to the language model that is being used, is shown to the user. If TiMBL returns multiple possible words (which is almost always the case), the word with the highest confidence value is chosen. When the user has already begun keying in the first letters of the next word, only words that start with exactly these letters are taken into consideration. For example, for both sentence \ref{ilike} and \ref{ilikes} Xxx will ask TiMBL to classify the 3-gream \emph{'\_ I like'}.

\begin{examples}
\item I like \label{ilike}
\item I like s \label{ilikes}
\end{examples}

Imagine that TiMBL returns the classes 'soup', 'icecream' and 'sandwiches', with a confidence of 0.5, 0.9 and 0.8 respectively. For sentence \ref{ilike}, 'icecream' will be chosen, because it has the highest confidence value. For \ref{ilikes}, however, the 's' has already been given, which rules out 'icecream'. From the words that are still possible 'sandwich' has the highest confidence value, so that is the one that Xxx will show the user.
%Confidence uitleggen

\textbf{Context-insensitive} modules work with a frequency list based on the training material. Xxx looks at the word that is being keyed in, and returns the most frequent word that matches with what has been given so far. For sentence \ref{ilike}, this will be the most frequent word on the list (because nothing has been given so far), for \ref{ilikes} this will be the most frequent word that starts with an s.

Xxx's modules can be concatenated. This means that a second modules takes over once the first module no longer has suggestions, the third module takes over once the second no longer has suggestions, etc. For all experiments described here, only two modules have been used: a context-sensitive one followed by a context-insensitive one, both based on the same material.

\begin{table*}[t]
\begin{tabular}{l|*{17}{l}}
Twitter user&1&2&3&4&5&6&7&8&9&10&11&12&13&14&15&16\\
\hline
Idiolect CKS&27&33&25&26&24&30&30&24&38&21&34&25&30&25&22&26\\
Idiolect SKKS&31&37&28&28&28&34&33&27&44&24&38&26&33&29&26&29\\
\hline
General model CKS&21&20\\
General model SKKS&21&20\\
\hline
Sociolect 1 CKS&36&36\\
Sociolect 1 SKKS&36&39\\
\hline
Sociolect 2 CKS&33&33\\
Sociolect 2 SKKS&37&37\\
\end{tabular}
\caption{Percentages of the for Twitter users 1-16, using 4 different language models.}
\label{result1}
\end{table*}

\begin{table*}[t]
\begin{tabular}{l|*{17}{l}}
Twitter user&17&18&19&20&21&22&23&24&25&26&27&28&29&30&31&32&33\\
\hline
Idiolect CKS&26&37&27&27&40&27&23&23&26&26&24&25&37&31&27&50&26\\
Idiolect SKKS&30&41&31&31&48&31&26&27&30&28&27&28&42&35&30&53&30\\
\hline
General model CKS\\
General model SKKS\\
\hline
Sociolect 1 CKS\\
Sociolect 1 SKKS\\
\hline
Sociolect 2 CKS\\
Sociolect 2 SKKS\\
\end{tabular}
\caption{Percentages of the for Twitter users 17-33, using 4 different language models.}
\label{result2}
\end{table*}

\begin{table*}[t]
\begin{tabular}{l|*{17}{l}}
Twitter user&34&35&36&37&38&39&40&41&42&43&44&45&46&47&48&49&50\\
\hline
Idiolect CKS&19&23&20&21&24&29&21&36&24&29&28&29&20&26&45&22&27\\
Idiolect SKKS&23&27&23&30&27&32&28&39&28&35&32&33&23&30&51&25&30\\
\hline
General model CKS\\
General model SKKS\\
\hline
Sociolect 1 CKS\\
Sociolect 1 SKKS\\
\hline
Sociolect 2 CKS\\
Sociolect 2 SKKS\\
\end{tabular}
\label{result3}
\caption{Percentages of the for Twitter users 33-50, using 4 different language models.}
\end{table*}


\subsection{Training and test material: tweets}

% hoe waren de users geselecteerd? Zo random mogelijk? Met bepaalde drempels?
By communicating with the Twitter API we were able to harvest all tweets of a manually created set of Dutch Twitter users from January until April 2013, and save every tweet they produced. Retweets were excluded.
%Meer tweets

To capture the networks of these users as extensively as possible, we extended our set, repeatedly at 30-minute intervals, with (1) the user most of the tweets in our collection referred to and (2) the user that was most referred to in general. This way were able to add users that had (1) a lot of connections to the people we were already following and (2) users at least one of the people we were following communicated with a lot. We limited ourselves to Dutch tweets.
% hoe? met het Twitter language ID?

In terms of amount of Tweets, the Twitter users show a Zipfian curve [[ref]]  [[uitzoeken of dat echt zo is!]]; a very small group produced a large amount of tweets and vice versa. For that reason, we limited ourselves to the 50 most frequent tweeters in the experiments described here. The amount of words ranged from 480.000 for the most frequent tweeter to 200.000 for the least frequent tweeter.

For each of these 50 users, a list of 'friends' is created. A Twitter user is considered a friend of another Twitter user when both users tweeted to the other person at least three times. We have chosen this number because a total of six tweets from both sides makes it likely that both Twitter users know each other. 

As a background corpus, a random selection of the blogs, emails and Wikipedia articles from the SoNaR corpus for written Dutch [[ref]] was used. These texts were chosen because they were believed to be neither very formal nor very informal. The control corpus consisted of 55,212,868 words.

\subsection{Evaluation measures}
The prestations of our text prediction system were measured in 'percentage of keystrokes saved'. We calculate these in two different ways:
\begin{itemize}
\item By only showing the user one prediction at a time, which could be accepted with the spacebar or a punctuation mark, or could be rejected with the tab key. As noted by \newcite{vandenbosch11}, for many devices and applications it might not be possible or practical for the user to inspect a list of multiple suggestions. The percentage of keystrokes that could have been saved when only one prediction was shown will be referred to as 'Classical Keystrokes Saved' (CKS).
\item The popular smartphone application SwiftKey always shows the user three predictions, which seem to be (1) what the user has keyed in so far, (2) the most likely prediction and (3) the second most likely prediction. In case the user has not yet started typing the next word, option (1) is replaced by the third most likely prediction. Because smartphones use a touchscreen as an input device, no extra keystrokes are needed. The percentage of keystrokes that could have been saved when two (and sometimes three) predictions were shown will be referred to as 'SwiftKey Keystrokes Saved' (SKKS).
\end{itemize}

\subsection{Procedure}
For each of the 50 selected Twitter users, the 10\% most recent Tweets were removed and used for testing purposes. It was tested how many keystrokes could have been saved on the basis of four language models: the \textbf{general model}, which was based on written texts by various authors, the \textbf{idiolect}, which was based on the 90\% remaining tweets from the same author, \textbf{sociolect 1}, which was based on 90\% of the remaining tweets from the same author \emph{and} the tweets of his/her friends, and \textbf{sociolect 2}, which was based which was based on 90\% of the remaining tweets from the same author and the tweets of the friends of a random other Twitter user. Sociolect 2 was created as a control model for sociolect 1, so we can tell to what extent any improvements of adding tweets can be attributed to the fact that these are tweets from friends.

\section{Results}
Table 1, 2 and 3 show the results for each of the 50 users. The users are ordered from most material to least material. A few things to note:

\begin{itemize}
\item Despite the general model having roughly 200 times as much training material, for every user the language model based on his/her own tweets performed much better.
\item For all users, adding tweets produced by others improves the results.
\item In almost all cases, adding tweets produced by friends (sociolect1) has better results than tweets by random twitter users (sociolect 2).
\item There are a lot of individual differences. For some users it is possible to correctly predict [[x]]\% of the characters, while for others we could only predict [[y]]\%.
\end{itemize}

\section{Conclusion}
In this paper we investigated whether our text prediction system could save more keystrokes in a text when it was trained on texts by the same author (question 1), using early tweets to predict more recent tweets. This turned out to be the case: despite the general model having roughly 200 times as much training material, the language models based on early tweets performed much better. Adding tweets by other Twitter users further improved the results (which is in line with \cite{verberne12}), but in most cases language models based on tweets by people related to the author somehow performed best (question 2).

\bibliographystyle{acl}
\bibliography{wesseltest}

\end{document}